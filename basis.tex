\section{学术论文排版}
\subsection{论文模板使用}

\begin{frame}{模板是什么?}
  \begin{itemize}
    \item 模板
      \begin{itemize}
        \item 已经设计好的格式框架
        \item 好的模板:使用户专注于内容
        \item 不应将时间花费在调整框架上
      \end{itemize}
    \item 再提 Office 和 Word
      \begin{itemize}
        \item 很少有人会有意识地在 Word 中使用模板
        \item 定义自己的标题?定义自己的列表?定义自己的段落样式?
        \item 自动化,还是手工调?
        \item 经常被折腾的精疲力竭
        \item 学习 \LaTeX{} 能帮助自己更好科学地使用 word
      \end{itemize}
  \end{itemize}
\end{frame}

\begin{frame}{论文排版}
  \begin{itemize}
    \item 获取模板
      \begin{itemize}
        \item 随发行版自带、手动网络下载
        \item 模板文档类 \texttt{.cls} 文件
        \item 示例 \texttt{.tex} 文件
      \end{itemize}
    \item 编辑 \texttt{.tex} 文件:添加用户内容
    \item 编译:生成 PDF 文档
  \end{itemize}
\end{frame}

\begin{frame}[fragile]{论文排版举例}
  \begin{exampleblock}{IEEE 期刊论文}
    \begin{itemize}
      \item 获取模板:已随发行版自带
        \begin{itemize}
          \item 在安装目录 \verb|<prefix>\texlive\2015\texmf-dist\doc\latex\IEEEtran|
            下找到 \verb|bare_jrnl.tex|
          \item 复制到某个文件夹(比如个人存论文的目录)
        \end{itemize}
      \item 编辑 \verb|bare_jrnl.tex| 文件 (英文模板:不支持中文)
      \item 编译
        \begin{itemize}
          \item 英文文献:XeLaTeX、PDFLaTeX 编译均可
        \end{itemize}
    \end{itemize}
  \end{exampleblock}
\end{frame}

\subsection{\LaTeX{} 排版入门}

\begin{frame}[fragile]{文件结构}
  \lstset{language=[LaTeX]TeX}
  \begin{lstlisting}[basicstyle=\ttfamily]
\documentclass[a4paper]{article}
% 文档类型,例如 article,[]内是选项,比如 a4paper 设置为 A4 纸
% 这里开始是导言区
\usepackage{graphicx} % 引用宏包
\graphicspath{{fig/}} % 设置
% 导言区到此为止
\begin{document}
这里开始是正文
\end{document}
  \end{lstlisting}
\end{frame}

\begin{frame}[fragile]{\LaTeX{}命令}
  \framesubtitle{\emph{宏} (Macro)、或者\emph{控制序列} (control sequence)}
\begin{itemize}
\item 简单命令
  \begin{itemize}
    \item \verb|\命令|\hspace{2em}
    \verb|{\songti 中国人民解放军}| ~$\Rightarrow$ {\songti 中国人民解放军}
  \item \verb|\命令[可选参数]{必选参数}|\\
\verb|section[精简标题]{这个题目实在太长了放到目录里面不太好看}|\\
$\Rightarrow$ {\heiti 1.1 \hspace{1em} \songti 这个题目实在太长了放到目录里面不太好看}
  \end{itemize}
\item 环境
  \begin{columns}[c]
  \begin{column}{0.45\textwidth}
    \begin{lstlisting}[basicstyle=\ttfamily]
\begin{equation*}
  a^2-b^2=(a+b)(a-b)
\end{equation*}
\end{lstlisting}
\end{column}\hspace{1em}
  \begin{column}{0.45\textwidth}
$ a^2-b^2=(a+b)(a-b)$
\end{column}
  \end{columns}
\end{itemize}
\end{frame}

\begin{frame}[fragile]{\LaTeX{} 常用命令}
  \begin{exampleblock}{命令}
\centering
\footnotesize
  \begin{tabular}{llll}
    \cmd{chapter} & \cmd{section} & \cmd{subsection} & \cmd{paragraph} \\
    章 & 节 & 小节 & 带题头段落 \\\hline
    \cmd{centering} & \cmd{emph} & \cmd{verb} & \cmd{url} \\
   居中对齐         &  强调      & 原样输出   & 超链接 \\\hline
  \cmd{footnote} & \cmd{item} & \cmd{caption} & \cmd{includegraphics} \\
   脚注 & 列表条目 & 标题 & 插入图片 \\\hline
  \cmd{label} & \cmd{cite} & \cmd{ref} \\
  标号 & 引用参考文献 & 引用图表公式等\\\hline
  \end{tabular}
\end{exampleblock}
\end{frame}
\begin{frame}[fragile]{\LaTeX{} 常用命令}
\begin{exampleblock}{环境}
\centering
\footnotesize
\begin{tabular}{lll}
  \env{table} & \env{figure} & \env{equation}\\
  表格 & 图片 & 公式 \\\hline
  \env{itemize} & \env{enumerate} & \env{description}\\
  无编号列表 & 编号列表 & 描述 \\\hline
\end{tabular}
\end{exampleblock}
\end{frame}
% 
\begin{frame}{\LaTeX{}命令举例}
\cmdxmp{chapter}{前言}{\heiti 第 1 章\hspace{1em} 前言}
\cmdxmp{section[精简标题]}{这个题目实在太长了放到目录里面不太好看}{\heiti 1.1
  \hspace{1em} 这个题目实在太长了放到目录里面不太好看}
\cmdxmp{footnote}{我是可爱的脚注}{前方高能\footnote{我是可爱的脚注}}
\end{frame}

\begin{frame}[fragile]{\LaTeX{} 环境命令举例}
  \begin{minipage}{0.4\linewidth}
    \begin{lstlisting}[basicstyle=\ttfamily\small]
\begin{itemize}
  \item 一条
  \item 次条
  \item 这一条可以分为 ...
    \begin{itemize}
      \item 子一条
    \end{itemize}
\end{itemize}
\end{lstlisting}
  \end{minipage}\hspace{1.5cm}
  \begin{minipage}{0.4\linewidth}
\begin{itemize}
  \item 一条
  \item 次条
  \item 这一条可以分为 ...
    \begin{itemize}
      \item 子一条
    \end{itemize}
\end{itemize}
  \end{minipage}
\medskip

  \begin{minipage}{0.4\linewidth}
\begin{lstlisting}
\begin{enumerate}
  \item 一条
  \item 次条
  \item 再条
\end{enumerate}
\end{lstlisting}
  \end{minipage}\hspace{1.5cm}
  \begin{minipage}{0.4\linewidth}
\begin{enumerate}
  \item 一条
  \item 次条
  \item 再条
\end{enumerate}
  \end{minipage}
\end{frame}
% 

\begin{frame}[fragile]{\LaTeX{} 数学公式}

\begin{columns}
\begin{column}{.5\textwidth}
\begin{lstlisting}[basicstyle=\ttfamily\small]
$V = \frac{4}{3}\pi r^2$

$$V = \frac{4}{3}\pi r^2$$

\begin{equation}
\label{eq:vsphere}
V = \frac{4}{3}\pi r^2
\end{equation}
\end{lstlisting}
\end{column}

\begin{column}{.5\textwidth}
$V = \frac{4}{3}\pi r^2$

$$V = \frac{4}{3}\pi r^2$$

\begin{equation}
\label{eq:vsphere}
V = \frac{4}{3}\pi r^2
\end{equation}
\end{column}
\end{columns}

\end{frame}

\begin{frame}{\LaTeX{} 数学公式}
\begin{itemize}
\item 数学公式排版是 \LaTeX{} 的绝对强项
\item 数学排版需要进入数学模式
	\begin{itemize}
	\item 用单个美元符号(\$) 包围起来的内容是 {\bf 行内公式}
	\item 用两个美元符号(\$\$) 包围起来的是 {\bf 单行公式}
	\item 使用数学环境,例如 \texttt{equation} 环境内的公式会自动加上编号,
		\texttt{align} 环境用于多行公式(例如方程组)
	\end{itemize}
\item 运行 \texttt{texdoc symbols} 查看符号表
\end{itemize}
\end{frame}

\begin{frame}[fragile]{层次与目录生成}
\begin{columns}
\begin{column}{.6\textwidth}

\begin{lstlisting}[basicstyle=\ttfamily\small]
\tableofcontents % 这里是目录
\part{有监督学习}
\chapter{支持向量机}
\section{支持向量机简介}
\subsection{支持向量机的历史}
\subsubsection{支持向量机的诞生}
\paragraph{一些趣闻}
\subparagraph{第一个趣闻}
\end{lstlisting}
\end{column}
\begin{column}{.4\textwidth}
第一部分\quad 有监督学习\\
第一章\quad 支持向量机 \\
1. 支持向量机简介 \\
1.1 支持向量机的历史 \\
1.1.1 支持向量机的诞生 \\
一些趣闻  \\
第一个趣闻
\end{column}
\end{columns}

\end{frame}


\begin{frame}[fragile]{列表与枚举}
\begin{columns}
\begin{column}{.6\textwidth}

  \begin{lstlisting}[basicstyle=\ttfamily\small]
\item \LaTeX{} 好处都有啥
  \begin{description}
    \item[好用] 体验好才是真的好
    \item[好看] 强迫症的福音
    \item[开源] 众人拾柴火焰高
  \end{description}
\item 还有呢?
  \begin{itemize}
    \item 好处 1
    \item 好处 2
  \end{itemize}
\end{enumerate}
\end{lstlisting}
\end{column}
\begin{column}{.4\textwidth}
{\small
\begin{enumerate}
\item \LaTeX{} 好处都有啥
  \begin{description}
    \item[好用] 体验好才是真的好
    \item[好看] 治疗强迫症
    \item[开源] 众人拾柴火焰高
  \end{description}
\item 还有呢?
  \begin{itemize}
    \item 好处 1
    \item 好处 2
  \end{itemize}
\end{enumerate}
}
\end{column}
\end{columns}

\end{frame}


\begin{frame}[fragile]{交叉引用}
  \begin{itemize}
  \item 给对象命名:图片、表格、公式等\\
  \verb|\label{name}|
\item 引用对象\\
  \verb|\ref{name}|
  \end{itemize}
\bigskip

  \begin{minipage}{0.7\linewidth}
    \begin{lstlisting}
图书馆馆徽请参见图~\ref{fig:lib}。
\begin{figure}[htbp]
  \centering
  \includegraphics[height=.2\textheight]%
  {libicon.pdf}
  \caption{图书馆馆徽。}
  \label{fig:lib}
\end{figure}
\end{lstlisting}
  \end{minipage}\hfill
  \begin{minipage}{0.3\linewidth}\centering
    {\songti 图书馆馆徽请参见图~1。}\\[1em]
 \includegraphics[height=0.2\textheight]{libicon.pdf}\\
 {\footnotesize\heiti 图~1. 图书馆馆徽。}
  \end{minipage}
\end{frame}

\begin{frame}[fragile]{交叉引用}
  \begin{columns}
  \column{.6\textwidth}
  \begin{lstlisting}
\begin{table}[htbp]
   \caption{编号与含义}
   \label{tab:number}
   \centering
   \begin{tabular}{cl}
     \toprule
     编号 & 含义 \\
     \midrule
     1    & 第一 \\
     2    & 第二 \\
     \bottomrule
   \end{tabular}
\end{table}
公式~(\ref{eq:vsphere}) 中编号与含义
请参见表~\ref{tab:number}。
\end{lstlisting}
\column{.4\textwidth}
\centering 
{\small 表~1. 编号与含义}\\[2pt]
\begin{tabular}{cl}\toprule
编号 & 含义 \\\midrule
1 & 第一\\
2  & 第二\\\bottomrule
\end{tabular}\\[5pt]

\normalsize 公式~(\ref{eq:vsphere})编号与含义请参见表~1。
  \end{columns}
\end{frame}

\begin{frame}[fragile]{浮动体}
\begin{itemize}
\item 初学者最``捉摸不透''的特性之一
\item 图片和表格有时会很大,在插入的位置不一定放得下,因此需要浮动调整
\item 避免在文中使用「下图」「上图」的说法,而是使用图表的编号,例如 \verb|图~\ref{fig:fig1}| 。
\item \verb|\begin{figure}[<位置>] 图片 \end{figure}|
\begin{itemize}
\item 位置参数指定浮动体摆放的偏好
\item \verb|h| 当前位置(here), \verb|t| 顶部(top), \verb|b| 底部(bottom), \verb|p| 单独成页(p)
\end{itemize}
\end{itemize}
\end{frame}

%%% vim: set ts=2 sts=2 sw=2 isk+=\: et tw=80 cc=+1 formatoptions+=mM:
