\section{学术论文排版}
\subsection{论文模板使用}

\begin{frame}{模板是什么?}
  \begin{itemize}
    \item 模板
      \begin{itemize}
        \item 已经设计好的格式框架
        \item 好的模板:使用户专注于内容
        \item 不应将时间花费在调整框架上
      \end{itemize}
    \item 再提 Office 和 Word
      \begin{itemize}
        \item 很少有人会有意识地在 Word 中使用模板
        \item 定义自己的标题?定义自己的列表?定义自己的段落样式?
        \item 自动化,还是手工调?
        \item 经常被折腾的精疲力竭
        \item 写论文成了学习 Word,大量不必要的精力消耗
      \end{itemize}
  \end{itemize}
\end{frame}

\begin{frame}{论文排版}
  \begin{itemize}
    \item 获取模板
      \begin{itemize}
        \item 随发行版自带、手动网络下载
        \item 模板文档类 \texttt{.cls} 文件
        \item 示例 \texttt{.tex} 文件
      \end{itemize}
    \item 编辑 \texttt{.tex} 文件:添加用户内容
    \item 编译:生成 PDF 文档
  \end{itemize}
\end{frame}

\begin{frame}[fragile]{论文排版举例}
  \begin{exampleblock}{IEEE 期刊论文}
    \begin{itemize}
      \item 获取模板:已随发行版自带
        \begin{itemize}
          \item 在安装目录 \lstinline|<prefix>\texlive\2014\|
            下搜索 \lstinline|bare_jrnl.tex|
        \end{itemize}
      \item 编辑 \texttt{.tex} 文件
        \begin{itemize}
          \item 打开 \lstinline|bare_jrnl.tex|,另存到个人论文存放的文件夹
          \item 英文模板:不支持中文
        \end{itemize}
      \item 编译
        \begin{itemize}
          \item 英文文献:XeLaTeX、PDFLaTeX 编译均可
        \end{itemize}
    \end{itemize}
  \end{exampleblock}
\end{frame}

\subsection{\LaTeX{}常用命令}

\begin{frame}[fragile]{\LaTeX{}命令}
  \framesubtitle{\emph{宏} (Macro)、或者\emph{控制序列} (control sequence)}
\begin{itemize}
\item 简单命令
  \begin{itemize}
    \item \verb|\命令|\hspace{2em}
    \verb|{\songti 中国人民解放军}| ~$\Rightarrow$ {\songti 中国人民解放军}
  \item \verb|\命令[可选参数]{必选参数}|\\
\verb|section[精简标题]{这个题目实在太长了放到目录里面不太好看}|\\
$\Rightarrow$ {\heiti 1.1 \hspace{1em} \songti 这个题目实在太长了放到目录里面不太好看}
  \end{itemize}
\item 环境
  \begin{columns}[c]
  \begin{column}{0.45\textwidth}
\begin{lstlisting}[style=latex]
\begin{equation*}
  a^2-b^2=(a+b)(a-b)
\end{equation*}
\end{lstlisting}
\end{column}\hspace{1em}
  \begin{column}{0.45\textwidth}
$ a^2-b^2=(a+b)(a-b)$
\end{column}
  \end{columns}
\end{itemize}
\end{frame}

\begin{frame}[fragile]{\LaTeX{} 常用命令}
  \begin{exampleblock}{命令}
\centering
\footnotesize
  \begin{tabular}{llll}
    \cmd{chapter} & \cmd{section} & \cmd{subsection} & \cmd{paragraph} \\
    章 & 节 & 小节 & 带题头段落 \\\hline
    \cmd{centering} & \cmd{emph} & \cmd{verb} & \cmd{url} \\
   居中对齐         &  强调      & 原样输出   & 超链接 \\\hline
  \cmd{footnote} & \cmd{item} & \cmd{caption} & \cmd{includegraphics} \\
   脚注 & 列表条目 & 标题 & 插入图片 \\\hline
  \cmd{label} & \cmd{cite} & \cmd{ref} \\
  标号 & 引用参考文献 & 引用图表公式等\\\hline
  \end{tabular}
\end{exampleblock}
\end{frame}
\begin{frame}[fragile]{\LaTeX{} 常用命令}
\begin{exampleblock}{环境}
\centering
\footnotesize
\begin{tabular}{lll}
  \env{table} & \env{figure} & \env{equation}\\
  表格 & 图片 & 公式 \\\hline
  \env{itemize} & \env{enumerate} & \env{description}\\
  无编号列表 & 编号列表 & 描述 \\\hline
\end{tabular}
\end{exampleblock}
\end{frame}

\begin{frame}{\LaTeX{}命令举例}
\cmdxmp{chapter}{前言}{\heiti 第 1 章\hspace{1em} 前言}
\cmdxmp{section[精简标题]}{这个题目实在太长了放到目录里面不太好看}{\heiti 1.1
  \hspace{1em} 这个题目实在太长了放到目录里面不太好看}
\cmdxmp{footnote}{我是可爱的脚注}{前方高能\footnote{我是可爱的脚注}}
\end{frame}

\begin{frame}[fragile]{\LaTeX{} 环境命令举例}
  \begin{minipage}{0.4\linewidth}
\begin{lstlisting}[style=latex]
\begin{itemize}
  \item 一条
  \item 次条
  \item 再条
\end{itemize}
\end{lstlisting}
  \end{minipage}\hspace{1.5cm}
  \begin{minipage}{0.4\linewidth}
\begin{itemize}
  \item 一条
  \item 次条
  \item 再条
\end{itemize}
  \end{minipage}
\medskip

  \begin{minipage}{0.4\linewidth}
\begin{lstlisting}[style=latex]
\begin{enumerate}
  \item 一条
  \item 次条
  \item 再条
\end{enumerate}
\end{lstlisting}
  \end{minipage}\hspace{1.5cm}
  \begin{minipage}{0.4\linewidth}
\begin{enumerate}
  \item 一条
  \item 次条
  \item 再条
\end{enumerate}
  \end{minipage}
\end{frame}

\begin{frame}[fragile]{自动引用举例}
  \begin{itemize}
  \item 给对象命名:图片、表格、公式等\\
  \verb|\label{name}|
\item 引用对象\\
  \verb|\ref{name}|
  \end{itemize}
\bigskip

  \begin{minipage}{0.5\linewidth}
\begin{lstlisting}[style=latex,basicstyle=\tiny]
图书馆馆徽请参见图~\ref{fig:lib}。
\begin{figure}
  \centering
  \includegraphics[height=0.2\textheight]%
  {libicon.jpg}
  \caption{图书馆馆徽。}
  \label{fig:lib}
\end{figure}
\end{lstlisting}
  \end{minipage}\hfill
  \begin{minipage}{0.4\linewidth}\centering
    {\songti 图书馆馆徽请参见图~1。}\\[1em]
 \includegraphics[height=0.2\textheight]{libicon.jpg}\\
 {\footnotesize\heiti 图~1. 图书馆馆徽。}
  \end{minipage}
\end{frame}

\begin{frame}[fragile]{自动引用举例}
\begin{minipage}{0.5\linewidth}
  \begin{lstlisting}[style=latex,basicstyle=\tiny]
\begin{table}
   \centering
   \begin{tabular}{ll}\hline
     编号 & 含义 \\\hline
     1    & 第一 \\
     2    & 第二 \\\hline
   \end{tabular}
   \caption{编号与含义。}
   \label{tab:number}
\end{table}
编号与含义请参见表~\ref{tab:number}。
\end{lstlisting}
\end{minipage}
\begin{minipage}{0.4\linewidth}
\centering\small
 \begin{tabular}{ll}\hline
编号 & 含义 \\\hline
1 & 第一\\
2  & 第二\\\hline
\end{tabular}\\[5pt]
{\small 表~1. 编号与含义}\\[1em]
\normalsize 编号与含义请参见表~1。
\end{minipage}
\end{frame}

