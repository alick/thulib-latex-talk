\documentclass{beamer}
% Author: alick<alick9188@gmail.com>

% This file is modified from a solution template for:

% - Giving a talk on some subject.
% - The talk is between 15min and 45min long.
% - Style is ornate.

% Copyright 2004 by Till Tantau <tantau@users.sourceforge.net>.
%
% In principle, this file can be redistributed and/or modified under
% the terms of the GNU Public License, version 2.
%
% However, this file is supposed to be a template to be modified
% for your own needs. For this reason, if you use this file as a
% template and not specifically distribute it as part of a another
% package/program, I grant the extra permission to freely copy and
% modify this file as you see fit and even to delete this copyright
% notice.

\mode<presentation>
{
  \usetheme[secheader]{Boadilla}
  \usefonttheme[onlymath]{serif}
  \setbeamercovered{transparent=5}
  %\definecolor{fedorablue}{RGB}{60,110,180}
  %\definecolor{fedoradarkblue}{RGB}{41,65,114}
  %\definecolor{fedoradarkgrey}{RGB}{76,76,76}
  %\setbeamercolor*{palette primary}{fg=white,bg=fedorablue}
  %\setbeamercolor*{palette secondary}{fg=white,bg=fedoradarkblue}
  %\setbeamercolor*{palette tertiary}{fg=white,bg=fedorablue}
  %\setbeamercolor*{palette quaternary}{fg=white,bg=black}
}

\usepackage{mflogo} % for \MF, \MP
\usepackage{graphicx}
\graphicspath{{fig/}}
\usepackage{listings}
\usepackage{xspace}
\usepackage{amsmath}
\usepackage{wasysym} % for \smiley etc
\usepackage{cclicenses} % CC symbols
\usepackage{fontspec}
\usepackage[UTF8,nofonts]{ctex}
\usepackage{hologo}
\usepackage{hyperxmp}
\hypersetup{
pdfauthor={Alick Zhao},
pdfcopyright={Copyright (C) 2015 by Alick Zhao.
Licensed under CC-BY-SA 4.0. Some rights reserved.},
pdflicenseurl={http://creativecommons.org/licenses/by-sa/4.0/},
}

% For tipa to work.
\newfontfamily\useTIPAfont{Times New Roman}

% xeCJK conf setup
\punctstyle{kaiming}
\renewcommand\CJKfamilydefault{\CJKsfdefault} % for slides

\setCJKmainfont[BoldFont={WenQuanYi Micro Hei},
ItalicFont={AR PL UKai CN}]{AR PL UMing CN}
\setCJKsansfont{WenQuanYi Micro Hei}
\setCJKmonofont{WenQuanYi Micro Hei Mono}

\setCJKfamilyfont{zhsong}{AR PL UMing CN}
\setCJKfamilyfont{zhhei}{WenQuanYi Zen Hei}
\setCJKfamilyfont{zhkai}{AR PL UKai CN}

\newcommand*{\songti}{\CJKfamily{zhsong}} % 宋体
\newcommand*{\heiti}{\CJKfamily{zhhei}}   % 黑体
\newcommand*{\kaishu}{\CJKfamily{zhkai}}  % 楷书

\renewcommand\tablename{表格}

\newcommand{\BibTeX}{\hologo{BibTeX}}
\newcommand{\XeTeX}{\hologo{XeTeX}}
\newcommand{\pdfTeX}{\hologo{pdfTeX}}
\newcommand{\beamer}{\textsc{beamer}}
\def\TeXLive{\TeX{} Live\xspace}
\let\TL=\TeXLive

\title
{如何使用 \LaTeX 排版论文}

\author[alick9188] % (optional, use only with lots of authors)
{赵涛\\ \texttt{alick9188@gmail.com}}

\institute[GitHub] % (optional, but mostly needed)
{
  清华大学电子系网络融合实验室
}
% - Use the \inst command only if there are several affiliations.
% - Keep it simple, no one is interested in your street address.

\date[图书馆专题培训讲座] % (optional)
{\today}

\subject{LaTeX, paper, ThuThesis}

% Delete this, if you do not want the table of contents to pop up at
% the beginning of each subsection:
\AtBeginSubsection[]
{
  \begin{frame}<beamer>{Outline}
    \tableofcontents[currentsection,currentsubsection]
  \end{frame}
}


% If you wish to uncover everything in a step-wise fashion, uncomment
% the following command:

%\beamerdefaultoverlayspecification{<+->}

\lstset{basicstyle=\ttfamily,breaklines=true}
\hypersetup{
%pdfpagemode=FullScreen,
}

%\logo{\includegraphics[height=.1\textheight]{Logo_fedoralogo.png}}

\begin{document}

\begin{frame}
  \titlepage
\end{frame}

\begin{frame}{Outline}
  \tableofcontents
  % You might wish to add the option [pausesections]
\end{frame}


% Since this a solution template for a generic talk, very little can
% be said about how it should be structured. However, the talk length
% of between 15min and 45min and the theme suggest that you stick to
% the following rules:

% - Exactly two or three sections (other than the summary).
% - At *most* three subsections per section.
% - Talk about 30s to 2min per frame. So there should be between about
%   15 and 30 frames, all told.

\section{简介}

\subsection{\TeX 和它的朋友们}

\begin{frame}[fragile]{\TeX 是什么?}
  \begin{itemize}
    \item 生成精美图书的排版系统
    \item 最初由 高德纳 (Donald E.~Knuth) 于 1978 年开发
    \item 可以免费且自由使用
    \item 最新版本为 \TeX\ 3.14159265
    \item 漂亮、美观、稳定、通用
    \item 尤其擅长数学公式排版
      \begin{equation}
        \mathcal{F}(\xi)=\int_{-\infty}^{\infty}
        f(x)\mathrm{e}^{-\mathrm{j}2\pi \xi x}\,\mathrm{d}x
      \end{equation}
    \item 读作 \textipa{/'tEx/} 或 \textipa{/'tEk/}
  \end{itemize}
\end{frame}

\begin{frame}{\LaTeX 是什么?}
  \begin{itemize}
    \item \TeX 用户接口不够友好,使用不便
    \item Leslie Lamport 开发 \LaTeX 降低使用门槛
    \item 核心仍是 \TeX
    \item 极其丰富的宏包(package),提供扩展功能
    \item 广泛用于学术界,期刊会议论文模板
    \item 大学学位论文模板,如 ThuThesis
    \item 当前版本 \LaTeXe
    %\item 读作 \textipa{/'ltEx/} 或 \textipa{/'tEk/}
  \end{itemize}
\end{frame}

\begin{frame}{\TeX 和它的朋友们}
  \begin{block}{\TeX\ 系统组成}
    \begin{itemize}
      \item 核心:宏语言
      \item 格式:Plain \TeX, \alert{\LaTeX} 等
      \item 引擎:tex, latex, pdf(la)tex, \alert{xe(la)tex} 等
      \item 发行版(套件):\alert{\TL}, MiK\TeX, Mac\TeX 等
    \end{itemize}
  \end{block}
\end{frame}

\subsection{安装(以 \TL 为例)}

\begin{frame}{\TL 简介}
  \begin{itemize}
    \item \TeX\ 发行版:相关引擎、宏包、文档的集合
    \item 由 \TeX 用户组 (TUG) 于 1996 年开始开发
    \item 跨平台支持:Linux, Windows, Mac OS X (Mac\TeX)
    \item 每年一个新版本发布,当前 \TL 2014
  \end{itemize}
\end{frame}

\begin{frame}[fragile]
  \frametitle{网络安装}
  \begin{itemize}
    \item 从 CTAN 镜像下载安装包(.zip 或 .tar.gz 格式)
(和相应的校验文件,以 .sha256 结尾)
\begin{itemize} % several mirror url
  \item 清华镜像 \url{http://mirrors.tuna.tsinghua.edu.cn/CTAN/systems/texlive/tlnet/}
  \item CTeX 镜像 \url{http://ftp.ctex.org/mirrors/CTAN/systems/texlive/tlnet/}
  \item 更多可见 \url{http://mirror.ctan.org/README.mirrors}
\end{itemize}

\item 可选步骤:校验安装包
\begin{lstlisting}
$ LANG=C sha256sum --check install-tl-unx.tar.gz.sha256
install-tl-unx.tar.gz: OK
\end{lstlisting}

  \end{itemize}
\end{frame}

\begin{frame}[fragile]
  \frametitle{网络安装}
  \begin{itemize}
    \item Windows
      \begin{itemize}
        \item 双击 \lstinline|install-tl-windows.bat| 或
          \lstinline|install-tl-advanced.bat|
      \end{itemize}
    \item Linux
      \begin{itemize}
        \item 图形安装界面需要 Perl Tk 模块:\texttt{yum install
          perl-Tk}
      \begin{lstlisting}
sudo mkdir /usr/local/texlive
sudo chown yourname:yourname /usr/local/texlive
./install-tl -gui -repository http://mirrors.tuna.tsinghua.edu.cn/CTAN/systems/texlive/tlnet/
      \end{lstlisting}
      \end{itemize}
\item 截图\dots
\end{itemize}
\end{frame}

\begin{frame}
  \begin{figure}[h]
  \centering
\includegraphics[scale=0.35]{main-init.png}
  \end{figure}
\end{frame}

\begin{frame}
  \begin{figure}[h]
  \centering
\includegraphics[scale=0.4]{标准安装.png}
  \end{figure}
\end{frame}

\begin{frame}
  \begin{figure}[h]
  \centering
\includegraphics[scale=0.4]{语言集合.png}
  \end{figure}
\end{frame}

\begin{frame}
  \begin{figure}[h]
  \centering
\includegraphics[scale=0.35]{main-customed.png}
  \end{figure}
\end{frame}

% 安装进行中
\begin{frame}
  \begin{figure}[h]
  \centering
\includegraphics[scale=0.45]{term.png}
  \end{figure}
\end{frame}

% 安装完成
\begin{frame}
  \begin{figure}[h]
  \centering
\includegraphics[scale=0.5]{安装过程.png}
  \end{figure}
\end{frame}

\begin{frame}[fragile]
  \frametitle{网络安装后配置(仅 Linux)}
\begin{itemize}
  \item
    添加环境变量到 \nolinkurl{~/.bash_profile} 文件:
    \begin{lstlisting}
export PATH=/usr/local/texlive/2011/bin/x86_64-linux:$PATH
export MANPATH=/usr/local/texlive/2011/texmf/doc/man:$MANPATH
export INFOPATH=/usr/local/texlive/2011/texmf/doc/info:$INFOPATH
    \end{lstlisting}

  \item
打开 \TeXLive 指南中文版 ``texlive-zh-cn.pdf'',
关注第 3.4 节
  \begin{lstlisting}
texdoc texlive-zh
  \end{lstlisting}

\end{itemize}
\end{frame}

\begin{frame}[fragile]
  \frametitle{网络安装后配置(仅 Linux)}
  \begin{itemize}
\item
\XeTeX\ 系统字体配置
\begin{lstlisting}
cp /usr/local/texlive/2011/texmf-var/fonts/conf/texlive-fontconfig.conf /etc/fonts/conf.d/09-texlive.conf
fc-cache -fsv
\end{lstlisting}

\end{itemize}
\end{frame}

\lstdefinestyle{latex}{
language={[LaTeX]TeX},
frame=single,
escapeinside=``,
keywordstyle=\color{red!70},
}

\begin{frame}
  \frametitle{网络安装后测试}
  \framesubtitle{English 测试}

  \begin{exampleblock}{使用已安装的示例文件}
    \begin{itemize}
      \item \texttt{latex sample2e.tex \#} .tex $\rightarrow$ .dvi (device independent)

      \texttt{xdvi sample2e.dvi \#} also try dvipdf sample2e.dvi
      \item try \texttt{pdflatex sample2e} directly
      \item \texttt{xetex opentype-info.tex \#} test of xetex's OpenType support
    \end{itemize}

  \end{exampleblock}
\end{frame}

\begin{frame}[fragile]
  \frametitle{网络安装后测试}
  \framesubtitle{中文测试 ``test-chinese.tex''}
  \begin{columns}[t]
    \begin{column}{.45\textwidth}
\begin{lstlisting}[style=latex]
\documentclass{ctexart}
\setCJKmainfont{AR PL UMing CN}
\begin{document}
\TeX{}`你好!`
\end{document}
\end{lstlisting}
\end{column}
    \begin{column}{.45\textwidth}
  \begin{lstlisting}
xelatex test-chinese
evince test-chinese.pdf
  \end{lstlisting}
  \fbox{\textrm \TeX{}\songti 你好!}
\end{column}
\end{columns}

\end{frame}

\section{论文排版}

\begin{frame}{论文排版(以 IEEEtran 为例)}
  \begin{itemize}
    \item 获取模板 (.cls 文件、示例 .tex 文件)
    \item 编辑 .tex 文件
    \item 编译
  \end{itemize}
\end{frame}

%\subsection{获取帮助}
%\begin{frame}{利用文档}
  %\begin{itemize}
    %\item tlmgr
      %\begin{itemize}
        %\item \texttt{tlmgr gui}
        %\item \texttt{man tlmgr}
      %\end{itemize}

    %\item texdoc
      %\begin{itemize}
        %\item e.g. \texttt{texdoc xecjk},
          %\texttt{texdoc mathmode}, and \texttt{texdoc symbols}
        %\item \texttt{texdoctk} --- GUI
      %\end{itemize}
  %\end{itemize}
%\end{frame}

\begin{frame}{\TeX\ 社区}
  \begin{columns}[c]
    \begin{column}{.45\textwidth}
      \begin{itemize}
        \item BBS
          \begin{itemize}
            \item \href{http://www.newsmth.net/nForum/board/TeX}{水木
              社区 TeX 版}
            \item \href{http://bbs.ctex.org/}{bbs.ctex.org}
          \end{itemize}
        \item 邮件列表
          \begin{itemize}
            \item \href{http://tug.org/mailman/listinfo/tex-live}{\TL}
            \item \href{http://www.linux.cz/pipermail/texlive/}{Fedora \TL
              Packaging}
          \end{itemize}
        \item \href{http://www.tex.ac.uk/cgi-bin/texfaq2html}{UK FAQ}
        \item \href{http://justfuckinggoogleit.com/}{Google}
        \item TeX StackExchange
      \end{itemize}
    \end{column}
    \begin{column}{.45\textwidth}
      \includegraphics[width=\textwidth]{TFZsuperellipse-crop.pdf}
    \end{column}
  \end{columns}
\end{frame}

\section*{总结}

\section*{附录}

\begin{frame}
  \begin{itemize}
    \item 本幻灯片基于:
      \begin{itemize}
        %\item templates: \url{http://github.com/alick9188/TeXLab}
        \item \url{http://github.com/alick9188/fad-texlive-talk}
        \item ThuThesis 使用向导
      \end{itemize}
    \item License: CC BY-SA 4.0 Unported \cc\ccby\ccsa
  \end{itemize}
\end{frame}

\begin{frame}
  \begin{center}
    {\LARGE Thanks!
    \bigskip

    Questions?}

  \end{center}
\end{frame}

\end{document}
%%% vim: set sw=2 isk+=\: et tw=70 formatoptions+=mM:
